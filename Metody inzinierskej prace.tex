% Metódy inžinierskej práce

\documentclass[10pt,twoside,slovak,a4paper]{article}

\usepackage[slovak]{babel}
%\usepackage[T1]{fontenc}
\usepackage[IL2]{fontenc} % lepšia sadzba písmena Ľ než v T1
\usepackage[utf8]{inputenc}
\usepackage{graphicx}
\usepackage{url} % príkaz \url na formátovanie URL
\usepackage{hyperref} % odkazy v texte budú aktívne (pri niektorých triedach dokumentov spôsobuje posun textu)

\usepackage{cite}
%\usepackage{times}

\pagestyle{headings}

\title{Odporúčacie systémy v Booking: Uľahčenie výberu v svete cestovania

\thanks{Semestrálny projekt v predmete Metódy inžinierskej práce, ak. rok 2024/25, vedenie: Richard Marko}} % meno a priezvisko vyučujúceho na cvičeniach

\author{Karolína Bujnová\\[2pt]
	{\small Slovenská technická univerzita v Bratislave}\\
	{\small Fakulta informatiky a informačných technológií}\\
	{\small \texttt{xbujnova@stuba.sk}}
	}

\date{\small 26. september 2024} % upravte



\begin{document}

\maketitle

\begin{abstract}
\ldots
\end{abstract}



\section{Úvod}
Každý z nás túži vidieť svet – spoznať, ako žijú ľudia v iných častiach sveta, ako fungujú iné kultúry, ochutnať nové jedlá, zažiť nové veci, zbaviť sa na chvíľu svojich problémov a nemyslieť na nepríjemné veci v našich životoch, alebo sa jednoducho zrelaxovať. Vďaka tomu sa turizmus stal veľmi populárnym, a s rastúcou popularitou rastie aj úsilie a investície vložené do rozvoja cestovného ruchu a podpory turizmu.
Svet ponúka nespočetné množstvo miest, ktoré sa oplatí navštíviť. Preto musíme zúžiť svoj výber, no nie vždy je to ľahké, najmä keď prihliadame na viacero faktorov, ako sú financie.
V takých prípadoch prichádzajú na rad odporúčacie systémy, ktoré používa väčšina (ak nie všetky) webové stránky a aplikácie. Tieto systémy nám ponúkajú personalizované informácie na základe našich predchádzajúcich vyhľadávaní a odporúčajú výhodné a relevantné možnosti, čím nám uľahčujú výber, šetria čas a predchádzajú informačnému preťaženiu.
Medzi najznámejšie aplikácie patrí Booking, ktorej hlavnou úlohou je pomôcť nám nájsť a zarezervovať hotel podľa našich predstáv. Avšak Booking ponúka oveľa viac. Poskytuje rôzne možnosti na výlety a aktivity, ktoré môžeme robiť v našom vybranom meste, pričom všetky tieto aktivity sú vyberané na základe našich vyhľadávaní a už absolvovaných výletov.
Odporúčacie systémy (OS) sú v tejto modernej dobe plnej technológií cenným nástrojom, ktorý nám uľahčuje rozhodovanie. Aké výhody teda ponúka Booking? Ako ovplyvňujú OS naše rozhodovanie? A ako sa spracúvajú naše informácie a histórií zadaných údajov a vyhľadávaní? Poďme sa na to spoločne pozrieť.
\section{Nejaká časť} \label{nejaka}

Z obr.~\ref{f:rozhod} je všetko jasné. 

\begin{figure*}[tbh]
\centering
%\includegraphics[scale=1.0]{diagram.pdf}
Aj text môže byť prezentovaný ako obrázok. Stane sa z neho označný plávajúci objekt. Po vytvorení diagramu zrušte znak \texttt{\%} pred príkazom \verb|\includegraphics| označte tento riadok ako komentár (tiež pomocou znaku \texttt{\%}).
\caption{Rozhodujúci argument.}
\label{f:rozhod}
\end{figure*}



\section{Iná časť} \label{ina}

Základným problémom je teda\ldots{} Najprv sa pozrieme na nejaké vysvetlenie (časť~\ref{ina:nejake}), a potom na ešte nejaké (časť~\ref{ina:nejake}).\footnote{Niekedy môžete potrebovať aj poznámku pod čiarou.}

Môže sa zdať, že problém vlastne nejestvuje\cite{Coplien:MPD}, ale bolo dokázané, že to tak nie je~\cite{Czarnecki:Staged, Czarnecki:Progress}. Napriek tomu, aj dnes na webe narazíme na všelijaké pochybné názory\cite{PLP-Framework}. Dôležité veci možno \emph{zdôrazniť kurzívou}.


\subsection{Nejaké vysvetlenie} \label{ina:nejake}

Niekedy treba uviesť zoznam:

\begin{itemize}
\item jedna vec
\item druhá vec
	\begin{itemize}
	\item x
	\item y
	\end{itemize}
\end{itemize}

Ten istý zoznam, len číslovaný:

\begin{enumerate}
\item jedna vec
\item druhá vec
	\begin{enumerate}
	\item x
	\item y
	\end{enumerate}
\end{enumerate}


\subsection{Ešte nejaké vysvetlenie} \label{ina:este}

\paragraph{Veľmi dôležitá poznámka.}
Niekedy je potrebné nadpisom označiť odsek. Text pokračuje hneď za nadpisom.



\section{Dôležitá časť} \label{dolezita}




\section{Ešte dôležitejšia časť} \label{dolezitejsia}




\section{Záver} \label{zaver} % prípadne iný variant názvu



%\acknowledgement{Ak niekomu chcete poďakovať\ldots}


% týmto sa generuje zoznam literatúry z obsahu súboru literatura.bib podľa toho, na čo sa v článku odkazujete
\bibliography{literatura}
\bibliographystyle{plain} % prípadne alpha, abbrv alebo hociktorý iný
\end{document}
